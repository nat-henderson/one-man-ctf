\documentclass[12pt]{article}
\title{Single-Player Capture The Flag}
\author{Evan Gallo, Benjamin Kaplan, Jason Kuster, and Nathan McKinley}
\begin{document}
\maketitle
\tableofcontents
\section{Introduction}
Capture The Flag, in a security context, is a competition used to train people
to identify and exploit vulnerabilities in an appliction. In the original game,
two teams were pitted against each other. The flag was a file on a computer,
and the last one to "touch" the file was currently in control of it.

More recently, a single-player variant has arisen. This variation is more of a
test of personal ability rather than a group competition. In this variant, the
player is given access to an account on a computer. Their job is to find a
vulnerability that will let them get the password to the next account down the
line. They win when they complete all of the challenges set before them.

\section{Our Game}
This project is a single-player implementation, focusing on vulnerabilities
common to networked code. The game is run on 4 different machines.
Specifically, we have designed this to run in Amazon's Elastic Cloud Compute
(EC2) service.

The game consists of 8 challenges spread out over 4 machines. Rather than using
passwords, all of the accounts on the machines are set up using private key
authentication. At the beginning of the game, the user is given a file that
contains encrypted versions of all of the private keys. Each key is encrypted
with a 20-character random password. The goal of the game is to read the text
file in the home directory of the final user. There is a README file in every
user's home directory that gives hints on how to proceed. 

\section{The Levels}

\subsection{Level 1}
\paragraph{Vulnerability} Path Injection
\paragraph{Summary} The goal is to read the password.txt file in the level02
home directory. The user is given access to a small program, as well as the
C code for the utility. The program uses setuid() to set the user id to level02 and then uses the system
function to call date.
\paragraph{Solution} Simply create your own executable program called "date"
that reads the password file and puts it on the path. Then, run the
printingdates program to get the password.

\subsection{Level 2}

\subsection{Level 3}

\subsection{Level 4}

\subsection{Level 5}

\subsection{Level 6}

\subection{Level 7}

\subsection{Level 8}
\end{document}
